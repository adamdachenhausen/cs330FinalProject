% 
% Sample latex source for a project report
%
% This is intended as an introduction to latex, but includes some
% advice about what you might want for the content of your paper
%
% Jim Teresco
% Williams College, Mount Holyoke College, Siena College, The College
% of Saint Rose
%
% Last modified: Mon Dec  4 19:03:24 EST 2017
%
%
% Modified by: Pat Baumgardner, Adam Dachenhausen, Shah Syed
%
%
% First, as you may have guessed, % is the way to define a latex 
% comment
%
% The first line tells latex you want for a default font size and that
% you want to create an ``article''
\documentclass[12pt]{article}
% extra packages to bring in
\usepackage{latexsym}
\usepackage{graphicx}      % extended graphics package
\usepackage{epsfig}        % wrapper for graphicx package
\usepackage{times}
\usepackage{url}
\usepackage{hyperef}
% set some margins, these can be defined as in, cm, pt
\setlength{\topmargin}{-0.5in}
\setlength{\textheight}{9in}
\setlength{\oddsidemargin}{0in}
\setlength{\evensidemargin}{0in}
\setlength{\textwidth}{6.5in}

% a few macros that might be useful -- any time we type \eg it expands
% to the italicized version defined here
\newcommand{\etal}{{\it et al}.$\:$}
\newcommand{\eg}{{\it e}.{\it g}.$\:$}
\newcommand{\cf}{{\it cf}.$\:$}
\newcommand{\ie}{{\it i}.{\it e}.$\:$}

%% to remove page numbers, uncomment this:
%% \pagestyle{empty}

%% Define single-space command
\newcommand{\singlespace}{
  \protect\renewcommand\baselinestretch{1.0}
  \protect\normalsize
}
% use this instead if you want to disable it completely:
%%\newcommand{\singlespace}{}

%% Define double-space command (really more like 1.5 spacing)
%% This is essential for rough drafts, and not a bad idea even for
%%  final submissions
\newcommand{\doublespace}{
  \protect\renewcommand\baselinestretch{1.5}
  \protect\normalsize
}
% use this instead if you want to disable it completely:
%%\newcommand{\doublespace}{}

% This tells latex we're done defining the preamble stuff and we're
% ready to start writing the document
\begin{document}

% this removes the date that is automatically placed in the title.
% comment it out if you want the date
\date{}

% the next few items define things to go onto the title section, like,
% well, the title, and the list of authors
\title{Java Based Memory Latency Simulator}

\author{Pat Baumgardner, Adam Dachenhausen, Shah Syed\\
Department of Computer Science\\
Siena College\\
Loudonville, NY  12211
}

% this tells latex that you're done setting up title stuff and that it
% should go ahead and generate the title here
\maketitle
% leave the page number off but for this page only
\thispagestyle{empty}

% Abstract!
\begin{abstract}

Insert abstract here.
  
\end{abstract}

% turn on double spacing
\doublespace

% Now we write the text of the paper, hopefully breaking it up into
% nice sections and subsections, using figures and tables as
% appropriate, and referring to those sections using labels instead of
% trying to number things by hand
\section{Overview}
\label{sec:overview}

The goal of this project was to create a computer memory latency simulator. 
For its ease of use, Java was chosen to program this project in. This 
simulator is designed to simulate data transfer from one system component 
to another. For example, from the hard drive disk, to the main memory. 

In Section~\ref{sec:memlate} we discuss abstractly how the memory latency 
simulator works. Section ~\ref{sec:build} we describe how to build and run the 
simulator. In section ~\ref{sec:expstats} we explain all the types of 
information collected from the simulator. Then, in ~\ref{sec:data} we 
present the actual data collected from the simulator, as well as actual 
real world data. Section ~\ref{sec:disc} goes on to explain any discrepencies 
found between our data and the real world collected data. Finally, in
~\ref{sec:conclusions} we summarize our results, as well as report what
we have learned, and suggest how this simulator could be better
improved or used. 

\section{Simulating memory latency}
\label{sec:memlate}

Insert detailed description of the code here. 
Also describe components and terms as you go. 

\section{Building and Running the Simulator}
\label{sec:build}

The simulator is written in Java, and therefore, you will need to have
Java installed to run it. See \href{https://www.java.com/en/download/}
{Java Download} for more.

\section{Explanation of Statistics Gathered}
\label{sec:expstats}

Explain every statistic. 
Also describe components and terms as you go. 

\section{Data Gathered}
\label{sec:data}

Insert all the data gathered by the simulator and from sources here.
Explain any terms as needed (only if not already). 

\section{disc}
\label{sec:disc}

Explain any discrepencies found here.

\section{Conclusions}
\label{sec:conclusions}

Summary, conclusions, discussion of the results, and ideas for future
work should be here.

Again, remember that writing is hard and writing well is very hard.  It
requires a long cycle of writing, reading, editing, rearranging,
rewriting, and so on.  If you are working in a group, have all group
members read and edit each others' sections.  Scientific writing
should be clear and concise.  If find a paragraph that says what can
be said in a sentence, cut it down to one sentence!

% it's nice to acknowledge any person or organization that helped out
% along the way.  The * after \section means this will not be assigned
% a section number
\section*{Acknowledgments}

The authors would like to thank all of our system administrators, past
and present, for installing the tools we needed on our systems and for
keeping those systems working smoothly while we completed this
project.

% Finally, the citations.  These lines tell latex what style to use
% (abbrv means to use first and middle initials instead of full names,
% among other things), and that your bibtex file is called
% references.bib
%
% bibtex alone should be enough to make anyone abandon any wysiwyg
% word processor and use latex for everything!
% 
% we'll also turn off double spacing for the citations
\singlespace
\bibliographystyle{abbrv}
\bibliography{references}

% tell latex we're done.  Anything beyond this line will be ignored.
\end{document}
