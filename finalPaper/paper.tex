% Jim Teresco
% Williams College, Mount Holyoke College, Siena College, The College
% of Saint Rose
%
% Last modified: Mon Dec  4 19:03:24 EST 2017
%
%
% Modified by: Pat Baumgardner, Adam Dachenhausen, Shah Syed
%
\documentclass[12pt]{article}
% extra packages to bring in
\usepackage{latexsym}
\usepackage{graphicx}      % extended graphics package
\usepackage{epsfig}        % wrapper for graphicx package
\usepackage{times}
\usepackage{url}
\usepackage{hyperref}
% set some margins, these can be defined as in, cm, pt
\setlength{\topmargin}{-0.5in}
\setlength{\textheight}{9in}
\setlength{\oddsidemargin}{0in}
\setlength{\evensidemargin}{0in}
\setlength{\textwidth}{6.5in}

% a few macros that might be useful -- any time we type \eg it expands
% to the italicized version defined here
\newcommand{\etal}{{\it et al}.$\:$}
\newcommand{\eg}{{\it e}.{\it g}.$\:$}
\newcommand{\cf}{{\it cf}.$\:$}
\newcommand{\ie}{{\it i}.{\it e}.$\:$}

%% to remove page numbers, uncomment this:
%% \pagestyle{empty}

%% Define single-space command
\newcommand{\singlespace}{
  \protect\renewcommand\baselinestretch{1.0}
  \protect\normalsize
}
% use this instead if you want to disable it completely:
%%\newcommand{\singlespace}{}

%% Define double-space command (really more like 1.5 spacing)
%% This is essential for rough drafts, and not a bad idea even for
%%  final submissions
\newcommand{\doublespace}{
  \protect\renewcommand\baselinestretch{1.5}
  \protect\normalsize
}
% use this instead if you want to disable it completely:
%%\newcommand{\doublespace}{}

% This tells latex we're done defining the preamble stuff and we're
% ready to start writing the document
\begin{document}

% this removes the date that is automatically placed in the title.
% comment it out if you want the date
\date{}

% the next few items define things to go onto the title section, like,
% well, the title, and the list of authors
\title{Java Based Memory Latency Simulator}

\author{Pat Baumgardner, Adam Dachenhausen, Shah Syed\\
Department of Computer Science\\
Siena College\\
Loudonville, NY  12211
}

% this tells latex that you're done setting up title stuff and that it
% should go ahead and generate the title here
\maketitle
% leave the page number off but for this page only
\thispagestyle{empty}

% Abstract!
\begin{abstract}

Insert abstract here.
  
\end{abstract}

% turn on double spacing
\doublespace

% Now we write the text of the paper, hopefully breaking it up into
% nice sections and subsections, using figures and tables as
% appropriate, and referring to those sections using labels instead of
% trying to number things by hand
\section{Overview}
\label{sec:overview}

The goal of this project was to create a computer memory latency simulator. 
For its ease of use, Java was chosen to program this project in. This 
simulator is designed to simulate data transfer from one system component 
to another. For example, from the hard drive disk, to the main memory. 

In Section~\ref{sec:memlate} we discuss abstractly how the memory latency 
simulator works, and why we built it. Section~\ref{sec:build} describes
how to build and run the simulator. In section~\ref{sec:expstats} we explain
all the types of  information collected from the simulator. Then, in section~\ref{sec:data}
we present the actual data collected from the simulator, as well as actual 
real world data. Section~\ref{sec:disc} goes on to explain any discrepencies 
found between our data and the real world collected data. Finally, in~\ref{sec:conclusions}
we summarize our results, as well as report what
we have learned, and suggest how this simulator could be better
improved or used. 

\section{Simulating memory latency}
\label{sec:memlate}

\subsection{Abstract Definition}
The goal of this project was to create a Java based simulator that will allow
semi-low access to artificial data so the user could gather statistics and trends
from moving this data around the artificial system. The simulator consists of four
primary parts: memSim, which is essentially the contoller; a CPU with one level of cache;
one or more sticks or RAM; and one or more hard drive disks.

Each component's (CPU, RAM, HDD), storage is represented with a byte array.
The user then has options to handle data on any component via the memSim terminal interface.

\subsection{Primary Audience}
Today, there are many levels of abstraction between the high-level user and their hardware.
We wanted to give a way to collect data at this low level of hardware, but not have
to be a electrical engineer working on the circuitry.

In our research, there is also very little public access on this type of data, so
it could be useful for researchers and students alike to use this simulator to collect
data for a variety of projects.

\section{Building and Running the Simulator}
\label{sec:build}

\subsection{Java}
The simulator is written in Java, and therefore, you will need to have
Java installed to run it. See \url{https://www.java.com/en/download/}
 for more.

\subsection{Aquiring Source Code}
To download the simulator code, go to \url{osfinal.dachenhausen.org} 
or
\url{https://github.com/adamdachenhausen/cs330FinalProject} 
and clone the repository. See \url{https://docs.github.com/en/free-pro-team@latest/github/creating-cloning-and-archiving-repositories/cloning-a-repository}
for more.

\subsection{Compiling}
\begin{itemize}

\item \textbf{Command Line}
  The source code includes two ways to compile the simulator.
  \begin{itemize}
  \item \textbf{Make}
  The simulator source code includes a Makefile, so if Make is installed, the command
  \begin{verbatim}make\end{verbatim}
  will compile the simulator.
  \item \textbf{Default}
  In the absence of Make, the default way to compile the simulator is to run  
  \begin{verbatim}javac *.java\end{verbatim}
  which will compile all of the Java files so they could be run.
  \end{itemize}
\item \textbf{Using an IDE}
  Given the multitude of IDEs that are available, please see your
  specific IDE's manual for how to compile and run the simulator.
  
\end{itemize}

\subsection{Licensing}
Before running the simulator, we would like to remind you that this
project is protected under the MIT License, so proceed at your own risk.

\subsection{Running the Simulator}
\begin{itemize}
\item \textbf{Command Line}
  Either way that the simulator was compiled, the command
\begin{verbatim}
  java memSim
\end{verbatim}
  will start the simulator
\item \textbf{IDE}
  Given the multitude of IDEs that are available, please see your
  specific IDE's manual for how to compile and run the simulator.
\end{itemize}
Upon running the simulator, you will be prompted for a variety of 
simulator parameters. Each of these is crucial, and cannot be left blank.
{\singlespace
\begin{verbatim}
How many Hard Drives would you like?
1
How many platters should each hard drive have?
4
And how big (in bytes) should each one be?
1024
How many RAM sticks would you like?
2
And how big (in bytes) should each one be?
128
Finally, how big (in bytes) would you like your CPU cache to be?
32
\end{verbatim}
}
After setting up the simulator, the components will each start up, and
you will be prompted to choose an option from the menu. You can choose to either
select based on the menu number, or the name.
{\singlespace
\begin{verbatim}
Java Based Memory Latency Simulator
Developed by Pat Baumgardner, Adam Dachenhausen, Shah Syed
Supported commands:
move
read
write
help
exit
For help with a specific command type 'help [command]'
\end{verbatim}
}

\section{Explanation of Data Gathered}
\label{sec:expstats}

Explain every statistics.

As we will be simulating the momeory simulator, we will be working on the virtual
components. These components are those linked directly with the memory usage, these will
include the central processing unit (CPU), the Hard Drive and the RAM sticks.
Throughout the simulation the CPU will be accessing the RAM and the Hard Drive to process
the data exxisting in both of these components. We are going to be timing the transfer
of data from both the Hard Drive and RAM.

The movement, reading and writing of the data to any of the components will be timed, which
will help us collect the major statistics we need here. The latency which is going to be the
major finding of the simulation is going to be calculted throught a stopwatch implemented
inside our program. As we keep clocking, the time is going to be a projected ratio of the time that will be taking
place on the basic level. Thus, the time will be reduced to the ratio fully suited for the
level of details being dealt with in the simulator.

The Memory, which is the core component of the simulator will be represented by array of bytes and
will be 

The variable we are going to have in the simulation are going to be the number of Cores in the CPU,
the number of hard drives, the size of hard drive in bytes, the number of RAM sticks, the size of
each of the RAM stick and the size of the CPU Cache. The values for each of these variables are
going to be input by the user of the simulator, before the program starts simulating the events.
The simulator is programmed in a way that different values for any these variables are going to effect
the statistics from a minor to a very significant way, depending upon the value of change and the
effectiveness of the change in the study.

CPU is going to have an option to be variable in number of cores, but for the path of the study being
conducted we are going to mantain the number of cores to just 1 throughout. Apart from the number
of cores, CPU will have a 


Also describe components and terms as you go. 

\section{Data Gathered}
\label{sec:data}

Insert all the data gathered by the simulator and from sources here.
Explain any terms as needed (only if not already). 

\section{Discrepencies}
\label{sec:disc}

Explain any discrepencies found here.

\section{Conclusions}
\label{sec:conclusions}

Insert conclusions here.

\singlespace
\bibliographystyle{abbrv}
\bibliography{references}

% tell latex we're done.  Anything beyond this line will be ignored.
\end{document}
