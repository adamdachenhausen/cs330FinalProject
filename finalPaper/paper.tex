% Jim Teresco
% Williams College, Mount Holyoke College, Siena College, The College
% of Saint Rose
%
% Last modified: Mon Dec  4 19:03:24 EST 2017
%
%
% Modified by: Pat Baumgardner, Adam Dachenhausen, Shah Syed
%
\documentclass[12pt]{article}
% extra packages to bring in
\usepackage{latexsym}
\usepackage{graphicx}      % extended graphics package
\usepackage{epsfig}        % wrapper for graphicx package
\usepackage{times}
\usepackage{url}
\usepackage{hyperref}
% set some margins, these can be defined as in, cm, pt
\setlength{\topmargin}{-0.5in}
\setlength{\textheight}{9in}
\setlength{\oddsidemargin}{0in}
\setlength{\evensidemargin}{0in}
\setlength{\textwidth}{6.5in}

% a few macros that might be useful -- any time we type \eg it expands
% to the italicized version defined here
\newcommand{\etal}{{\it et al}.$\:$}
\newcommand{\eg}{{\it e}.{\it g}.$\:$}
\newcommand{\cf}{{\it cf}.$\:$}
\newcommand{\ie}{{\it i}.{\it e}.$\:$}

%% to remove page numbers, uncomment this:
%% \pagestyle{empty}

%% Define single-space command
\newcommand{\singlespace}{
  \protect\renewcommand\baselinestretch{1.0}
  \protect\normalsize
}
% use this instead if you want to disable it completely:
%%\newcommand{\singlespace}{}

%% Define double-space command (really more like 1.5 spacing)
%% This is essential for rough drafts, and not a bad idea even for
%%  final submissions
\newcommand{\doublespace}{
  \protect\renewcommand\baselinestretch{1.5}
  \protect\normalsize
}
% use this instead if you want to disable it completely:
%%\newcommand{\doublespace}{}

% This tells latex we're done defining the preamble stuff and we're
% ready to start writing the document
\begin{document}

% this removes the date that is automatically placed in the title.
% comment it out if you want the date
\date{}

% the next few items define things to go onto the title section, like,
% well, the title, and the list of authors
\title{Java Based Memory Latency Simulator}

\author{Pat Baumgardner, Adam Dachenhausen, Shah Syed\\
Department of Computer Science\\
Siena College\\
Loudonville, NY  12211
}

% this tells latex that you're done setting up title stuff and that it
% should go ahead and generate the title here
\maketitle
% leave the page number off but for this page only
\thispagestyle{empty}

% Abstract!
\begin{abstract}

Insert abstract here.
  
\end{abstract}

% turn on double spacing
\doublespace

% Now we write the text of the paper, hopefully breaking it up into
% nice sections and subsections, using figures and tables as
% appropriate, and referring to those sections using labels instead of
% trying to number things by hand
\section{Overview}
\label{sec:overview}

The goal of this project was to create a computer memory latency simulator. 
For its ease of use, Java was chosen to program this project in. This 
simulator is designed to simulate data transfer from one system component 
to another. For example, from the hard drive disk, to the main memory. 

In Section~\ref{sec:memlate} we discuss abstractly how the memory latency 
simulator works. Section ~\ref{sec:build} we describe how to build and run the 
simulator. In section ~\ref{sec:expstats} we explain all the types of 
information collected from the simulator. Then, in ~\ref{sec:data} we 
present the actual data collected from the simulator, as well as actual 
real world data. Section ~\ref{sec:disc} goes on to explain any discrepencies 
found between our data and the real world collected data. Finally, in
~\ref{sec:conclusions} we summarize our results, as well as report what
we have learned, and suggest how this simulator could be better
improved or used. 

\section{Simulating memory latency}
\label{sec:memlate}

Insert detailed description of the code here. 
Also describe components and terms as you go. 

\section{Building and Running the Simulator}
\label{sec:build}

\subsection{Java}
The simulator is written in Java, and therefore, you will need to have
Java installed to run it. See \url{https://www.java.com/en/download/}
 for more.

\subsection{Aquiring Source Code}
To download the simulator code, go to \url{osfinal.dachenhausen.org} 
or
\url{https://github.com/adamdachenhausen/cs330FinalProject} 
and clone the repository. See \url{https://docs.github.com/en/free-pro-team@latest/github/creating-cloning-and-archiving-repositories/cloning-a-repository}
for more.

\subsection{Compiling}
\begin{itemize}

\item \textbf{Command Line}
  The source code includes two ways to compile the simulator.
  \begin{itemize}
  \item \textbf{Make}
  The simulator source code includes a Makefile, so if Make is installed, the command
  \begin{verbatim}make\end{verbatim}
  will compile the simulator.
  \item \textbf{Default}
  In the absence of Make, the default way to compile the simulator is to run  
  \begin{verbatim}javac *.java\end{verbatim}
  which will compile all of the Java files so they could be run.
  \end{itemize}
\item \textbf{Using an IDE}
  Given the multitude of IDEs that are available, please see your
  specific IDE's manual for how to compile and run the simulator.
  
\end{itemize}

\subsection{Licensing}
Before running the simulator, we would like to remind you that this
project is protected under the MIT License, so proceed at your own risk.

\subsection{Running the Simulator}
\begin{itemize}
\item \textbf{Command Line}
  Either way that the simulator was compiled, the command
\begin{verbatim}
  java memSim
\end{verbatim}
  will start the simulator
\item \textbf{IDE}
  Given the multitude of IDEs that are available, please see your
  specific IDE's manual for how to compile and run the simulator.
\end{itemize}
Upon running the simulator, you will be prompted for a variety of 
simulator parameters. Each of these is crucial, and cannot be left blank.
{\singlespace
\begin{verbatim}
How many Hard Drives would you like?
1
How many platters should each hard drive have?
4
And how big (in bytes) should each one be?
1024
How many RAM sticks would you like?
2
And how big (in bytes) should each one be?
128
Finally, how big (in bytes) would you like your CPU cache to be?
32
\end{verbatim}
}
After setting up the simulator, the components will each start up, and
you will be prompted to choose an option from the menu. You can choose to either
select based on the menu number, or the name.
{\singlespace
\begin{verbatim}
Java Based Memory Latency Simulator
Developed by Pat Baumgardner, Adam Dachenhausen, Shah Syed
Supported commands:
move
read
write
help
exit
For help with a specific command type 'help [command]'
\end{verbatim}
}

\section{Explanation of Data Gathered}
\label{sec:expstats}

Explain every statistic. 
Also describe components and terms as you go. 

\section{Data Gathered}
\label{sec:data}

Insert all the data gathered by the simulator and from sources here.
Explain any terms as needed (only if not already). 

\section{Discrepencies}
\label{sec:disc}

Explain any discrepencies found here.

\section{Conclusions}
\label{sec:conclusions}

Insert conclusions here.

\singlespace
\bibliographystyle{abbrv}
\bibliography{references}

% tell latex we're done.  Anything beyond this line will be ignored.
\end{document}
