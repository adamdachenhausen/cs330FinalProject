% Jim Teresco
% Williams College, Mount Holyoke College, Siena College, The College
% of Saint Rose
%
% Last modified: Mon Dec  4 19:03:24 EST 2017
%
%
% Modified by: Pat Baumgardner, Adam Dachenhausen, Shah Syed
%
\documentclass[12pt]{article}
% extra packages to bring in
\usepackage{latexsym}
\usepackage{graphicx}      % extended graphics package
\usepackage{epsfig}        % wrapper for graphicx package
\usepackage{times}
\usepackage{url}
\usepackage{hyperref}
% set some margins, these can be defined as in, cm, pt
\setlength{\topmargin}{-0.5in}
\setlength{\textheight}{9in}
\setlength{\oddsidemargin}{0in}
\setlength{\evensidemargin}{0in}
\setlength{\textwidth}{6.5in}

% a few macros that might be useful -- any time we type \eg it expands
% to the italicized version defined here
\newcommand{\etal}{{\it et al}.$\:$}
\newcommand{\eg}{{\it e}.{\it g}.$\:$}
\newcommand{\cf}{{\it cf}.$\:$}
\newcommand{\ie}{{\it i}.{\it e}.$\:$}

%% to remove page numbers, uncomment this:
%% \pagestyle{empty}

%% Define single-space command
\newcommand{\singlespace}{
  \protect\renewcommand\baselinestretch{1.0}
  \protect\normalsize
}
% use this instead if you want to disable it completely:
%%\newcommand{\singlespace}{}

%% Define double-space command (really more like 1.5 spacing)
%% This is essential for rough drafts, and not a bad idea even for
%%  final submissions
\newcommand{\doublespace}{
  \protect\renewcommand\baselinestretch{1.5}
  \protect\normalsize
}
% use this instead if you want to disable it completely:
%%\newcommand{\doublespace}{}

% This tells latex we're done defining the preamble stuff and we're
% ready to start writing the document
\begin{document}

% this removes the date that is automatically placed in the title.
% comment it out if you want the date
\date{}

% the next few items define things to go onto the title section, like,
% well, the title, and the list of authors
\title{Java Based Memory Latency Simulator}

\author{Pat Baumgardner, Adam Dachenhausen, Shah Syed\\
Department of Computer Science\\
Siena College\\
Loudonville, NY  12211
}

% this tells latex that you're done setting up title stuff and that it
% should go ahead and generate the title here
\maketitle
% leave the page number off but for this page only
\thispagestyle{empty}

% Abstract!
\begin{abstract}

The Java Based Memory Latency Simulator was developed to provide a way to
study how long it takes to move bytes of data around a common computer
system containing hard drive disks, random access memory, and a central
processing unit with cache memory. This project aims first to create a modular
system, that could be used to simulate data movement computer components, then
second to use that simulator and upgrade it to allow data to be collected about
data latency. We first focused on creating data structures that represented various
components of the computer, like a hard drive disk. Then we created a dual purpose
structure that both holds these components and interacts with the user. For this project,
we gathered data about the memory latency as an example of the modularity of the simulator.
  
\end{abstract}

% turn on double spacing
\doublespace

% Now we write the text of the paper, hopefully breaking it up into
% nice sections and subsections, using figures and tables as
% appropriate, and referring to those sections using labels instead of
% trying to number things by hand
\section{Overview}
\label{sec:overview}

The goal of this project was to create a computer memory latency simulator. 
Java was chosen as the programming language for this project for its ease of use and
it is what we, the authors are most comfortable with.
This simulator is designed to simulate data transfer from one system component 
to another. For example, from the hard drive disk, to the main memory. 

In Section~\ref{sec:memlate} we discuss abstractly how the memory latency 
simulator works, and why we built it. Section~\ref{sec:build} describes
how to build and run the simulator. In section~\ref{sec:expstats} we explain
all the types of information collected from the simulator. Then, in section~\ref{sec:data}
we present the actual data collected from the simulator, as well as actual 
real world data. Section~\ref{sec:disc} goes on to explain any discrepancies 
found between our data and the real world collected data. Finally, in~\ref{sec:conclusions}
we summarize our results, as well as report what
we have learned, and suggest how this simulator could be better
improved or used. 

\section{Simulating memory latency}
\label{sec:memlate}

\subsection{Abstract Definition}
The goal of this project was to create a Java based simulator that will allow
access to artificial data so the user could gather statistics and trends
from moving this data around the artificial system. The simulator allows simple operations
on blocks of data, not individual bytes. The simulator consists of four
primary parts: memSim, which is essentially the controller; a CPU with one level of cache;
one or more sticks or RAM; and one or more hard drive disks.

Each component's (CPU, RAM, HDD), storage is represented with a simple data structure called
a dataBlock, which represents 1024 bytes. The user then has options to handle data on any
component via the memSim terminal interface.

\subsection{Primary Audience}
Today, there are many levels of abstraction between the high-level user and their hardware.
We wanted to give a way to collect data at this low level of hardware, but not have
to be a electrical engineer working on the circuitry.

In our research, there is also very little public access on this type of data, so
it could be useful for researchers and students alike to use this simulator to collect
data for a variety of projects.

\section{Building and Running the Simulator}
\label{sec:build}

\subsection{Java}
The simulator is written in Java, and therefore, you will need to have
Java installed to run it. See \url{https://www.java.com/en/download/}
 for more.

\subsection{Acquiring Source Code}
To download the simulator code, go to \url{osfinal.dachenhausen.org} 
or
\url{https://github.com/adamdachenhausen/cs330FinalProject} 
and clone the repository. See \url{https://docs.github.com/en/free-pro-team@latest/github/creating-cloning-and-archiving-repositories/cloning-a-repository}
for more.

\subsection{Compiling}
\begin{itemize}

\item \textbf{Command Line}
  The source code includes two ways to compile the simulator.
  \begin{itemize}
  \item \textbf{Make}
  The simulator source code includes a Makefile, so if Make is installed, the command
  \begin{verbatim}make\end{verbatim}
  will compile the simulator.
  \item \textbf{Default}
  In the absence of Make, the default way to compile the simulator is to run  
  \begin{verbatim}javac *.java\end{verbatim}
  which will compile all of the Java files so they could be run.
  \end{itemize}
\item \textbf{Using an IDE}
  Given the multitude of IDEs that are available, please see your
  specific IDE's manual for how to compile and run the simulator.
  
\end{itemize}

\subsection{Licensing}
Before running the simulator, we would like to remind you that this
project is protected under the MIT License, so proceed at your own risk.

\subsection{Running the Simulator}
\begin{itemize}
\item \textbf{Command Line}
  Either way that the simulator was compiled, the command
\begin{verbatim}
  java memSim
\end{verbatim}
  will start the simulator
\item \textbf{IDE}
  Given the multitude of IDEs that are available, please see your
  specific IDE's manual for how to compile and run the simulator.
\end{itemize}
Upon running the simulator, you will be prompted for a variety of 
simulator parameters. Each of these is crucial, and cannot be left blank.

The first thing the simulator will ask you is
{\singlespace
\begin{verbatim}
Is this text blue? Y/n
\end{verbatim}
If your terminal supports ANSI colored text, then the question should be blue,
and you should respond Y. This helps point out important data.
If your terminal does not support the colored text, then you will see something like
{\singlespace
\begin{verbatim}
[34mIs this text blue? [0m Y/n
\end{verbatim}
in which case, if you respond n, the simulator will not use any ANSI text and
everything will print normally.

The simulator will then ask you whether you would like to LOAD or create a NEW system.

\subsection*{Load An Existing System}
The simulator will then show you three system specifications and a prompt, in which
you should choose a system by entering its system number.

\subsection*{Create A New System}
If you opt to create a new system, you will be presented with the prompt shown below,
which has sample data included. Note: very large numbers will make the simulator lag
the Java virtual machine, so please be careful when inputting this data.
{\singlespace
\begin{verbatim}
How many Hard Drives would you like?
1
And how big (in bytes) should each one be?
1024
How many RAM sticks would you like?
2
And how big (in bytes) should each one be?
128
Finally, how big (in bytes) would you like your CPU cache to be?
32
\end{verbatim}
}
After setting up the simulator, the components will each start up, and
you will be prompted to choose an option from the menu. You can choose to either
select based on the menu number, or the name.
{\singlespace
\begin{verbatim}
Java Based Memory Latency Simulator
Developed by Pat Baumgardner, Adam Dachenhausen, Shah Syed
Supported commands:
move
read
write
delete
help
exit
For help with a specific command type [command]
Type 'q' to quit the help menu
\end{verbatim}
}

In all actions, the simulator starts its timing only when user input is done,
so your data will not reflect your typing speed.

\subsection*{Reading Data}
Reading data is fairly simple, upon selecting read by entering its number into
the prompt, the simulator will show the following menu:
{\singlespace
\begin{verbatim}
Where would you like to read the data from?
1.      Hard Drive
2.      RAM
3.      CPU Cache
Please enter the number of your selection:
\end{verbatim}
}
You should then enter the number corresponding to your selection.
If you select a hard drive or a RAM stick, you will first get shown:
{\singlespace
\begin{verbatim}
Please choose a drive from 1 - 2
\end{verbatim}
}
Then, you will get shown that drive/stick's specific contents:
{\singlespace
\begin{verbatim}
****Data on this hard drive****
0: 1024 bytes big
2: 1024 bytes big
3: 1024 bytes big
*******************************
\end{verbatim}
}
You should then select the data that you wish to be read.

If you select the CPU cache, the entire contents of the cache will be read.

\subsection*{Writing Data}
Writing data is similar to reading data, as you will still be shown a prompt
of where you would like to write, and which specific drive/stick's contents to write to.
You will then be shown:
{\singlespace
\begin{verbatim}
How many bytes would you like to write?
\end{verbatim}
}
In which, you should enter the number of bytes to write. Note, if you enter a number greater
than the space remaining on the drive/stick, then the write will fail.

If you select the CPU cache, there is no limiter (other than that of Java) on maximum data
that can be written. The data structure is set up so that writing to cache is circular, so
the last n bytes are the ones stored in the cache, where n is the size of the cache.

\subsection*{Deleting Data}
Deleting data is very similar to reading data, as they are essentially the same operation.
In this case, every iteration through the stored data, the current index is set to null.

\subsection*{Moving Data}
A move is a combination of the three aforementioned commands. First, the simulator will prompt
like a read from where to read, and then delete the data from. Then the simulator will prompt
for where this data should be written. At this point the simulator will also make sure that
you do not try to read from one location, then write to that exact same location, ie. read
from drive 1, chunk 0, then try to write to drive 1, chunk 0. The simulator will re-prompt until
this is not the case. 
\section{Explanation of Data Gathered}
\label{sec:expstats}


The primary path of study through the simulator was to gather the statistics about the
events, latencies and delays that take place in the very low level of memory management.
Memory management as the name suggests is the act of managing memory in the low level of
a computer. Often represented by allocation of portions and freeing them once they are
done being used.

If we consider memory management as the super class, then the statistics that we gathered, such
as the latency of memory regulation are the one of the aspects of the memory management.
The latency is the delay or the time that it takes the data to be read from memory.
Since the movement of this data is bound to be within the speed of light,
the medium it is traveling in, this means that even though that speed is above human physical
comprehension but still there would be a delay, based on the size of the data.

We gathered the latency time, which is the time delay taking place in data transfer,reading and
writing at the lowest level of memory management. This was the primary statistic for us to consider
when building the simulator.

To represent this level of memory management the simulator was designed to hold some key components.
These components are those linked directly with the memory usage,including the
central processing unit (CPU), the Hard Drive and the RAM sticks.

The CPU further consists of cache, modern day powerful CPUS consisted of 2 level of caches, L1 and L2,
where L1 is very small compared to L2 in memory. L1 ranging from 2KB to 64KB, whereas the second level
is in the range of 256 KB to 2 MB. Cache is used for quicker access by the CPU of the data that it
frequently uses. Cache stores that data and thus decreasing the latency. However, in the simulator we
implemented a single level of cache with a variable size, which is input by the user.

Further down the hierarchy of memory management, we have the RAM, which is relatively slower to CPU but bigger
in size by quite some margin. The RAM, which translates to Random Access Memory is a volatile form of memory
built by several memory cells. There are two types of RAMs; DRAM and SRAM, even thought we did not distinguish
between them in our simulator. Given that the RAM is slower compared to the CPU(cache) it takes time to read
and move data off it. In the simulator we kept the size and the number of RAM sticks a variable, so
the user inputs the value for each. Since the RAM is volatile it loses all the data it has on it, once we
take the power off it. As soon as the power is turned on, the alot of important section of OS is read from
the hard drive and put onto the RAM so some important functions can be accessed quickly instead of reading
it off the hard drive which is way slower. This OS data along with other make up alot the RAM usage when the
computer is running. Running other process lead to more writing and reading off the RAM which leads to different
delays.

Finally comes the Hard Drive, which is the biggest form of permanent memory but the slowest one. It is
extremely large compared to both RAM and cache. Almost all the software including the OS is stored on
the HDD. HDDs have evolved in many ways over the course of times, making them quicker, but they have
remained slower compared to RAM and CPU always. In our simulator we have built an HDD on arrays of bytes,
just like the RAM and cache. 

The movement, reading and writing of the data to any of the components was timed, which helped
us to  collect the major statistics we needed  here. The latency which was  going to be the
major finding of the simulation was calculated through a stopwatch implemented
inside our program. As we kept clocking, the time was a projected ratio of the time that was supposed to
take place on the basic level. Thus, the time gathered was reduced to the ratio fully suited for the
level of details being dealt with in the simulator.

The Memory, which was the core component of the simulator and the primary part of each component
was represented by arrays of byte type. As the simulation went on, these arrays in each of the components
were modified depending upon the event taking place. This modification gave us the data for how the size
of data acts proportional to the latency and the delay in the transfer, writing and reading of data. The size
of data was one of the major field of the statistics being gathered. We studied upon the size of the
data as how much of variance does it brings to the clock.

As the variables we had in the simulation were the number of Cores in the CPU,
the number of hard drives, the size of hard drive in bytes, the number of RAM sticks, the size of
each of the RAM stick and the size of the CPU Cache, we kept a record of each combination. However,
even with the CPU having a variable number of cores, for the path of study being conducted, we maintained
it to be 1 throughout the study. The simulator was structured to take in a value for each of the variables,
except the number of cores. This led to new fields of data for statistics being gathered.
The size of cache in the CPU, which is a small form of memory and the quickest one out of RAM and HDD, was
varied to yield a different result, which went into the into the table of statistics too.
The same was done with the RAM, which relatively larger form of memory. We gathered how the data access
and transfer on RAM vary in the aspect of Time as with different sizes, and how slower or faster was the case
with the accessing, editing and transferring from and to the Hard drives.

The number of RAM sticks and the number of hard drive was also an important variable in deciding the latency,
the access time and the writing time, given that increasing the number of these components meant more
size but a different access time.

\section{Data Gathered}
\label{sec:data}


Insert all the data gathered by the simulator and from sources here.


Explain any terms as needed (only if not already). 

\section{Discrepancies}
\label{sec:disc}

Explain any discrepancies found here.

\section{Conclusions}
\label{sec:conclusions}

Insert conclusions here.

\singlespace
\bibliographystyle{abbrv}
\bibliography{references}

% tell latex we're done.  Anything beyond this line will be ignored.
\end{document}
